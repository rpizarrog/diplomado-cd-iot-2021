% Options for packages loaded elsewhere
\PassOptionsToPackage{unicode}{hyperref}
\PassOptionsToPackage{hyphens}{url}
%
\documentclass[
]{book}
\usepackage{amsmath,amssymb}
\usepackage{lmodern}
\usepackage{ifxetex,ifluatex}
\ifnum 0\ifxetex 1\fi\ifluatex 1\fi=0 % if pdftex
  \usepackage[T1]{fontenc}
  \usepackage[utf8]{inputenc}
  \usepackage{textcomp} % provide euro and other symbols
\else % if luatex or xetex
  \usepackage{unicode-math}
  \defaultfontfeatures{Scale=MatchLowercase}
  \defaultfontfeatures[\rmfamily]{Ligatures=TeX,Scale=1}
\fi
% Use upquote if available, for straight quotes in verbatim environments
\IfFileExists{upquote.sty}{\usepackage{upquote}}{}
\IfFileExists{microtype.sty}{% use microtype if available
  \usepackage[]{microtype}
  \UseMicrotypeSet[protrusion]{basicmath} % disable protrusion for tt fonts
}{}
\makeatletter
\@ifundefined{KOMAClassName}{% if non-KOMA class
  \IfFileExists{parskip.sty}{%
    \usepackage{parskip}
  }{% else
    \setlength{\parindent}{0pt}
    \setlength{\parskip}{6pt plus 2pt minus 1pt}}
}{% if KOMA class
  \KOMAoptions{parskip=half}}
\makeatother
\usepackage{xcolor}
\IfFileExists{xurl.sty}{\usepackage{xurl}}{} % add URL line breaks if available
\IfFileExists{bookmark.sty}{\usepackage{bookmark}}{\usepackage{hyperref}}
\hypersetup{
  pdftitle={Introducción a la Ciencia de los Datos},
  pdfauthor={Rubén Pizarro Gurrola},
  hidelinks,
  pdfcreator={LaTeX via pandoc}}
\urlstyle{same} % disable monospaced font for URLs
\usepackage{longtable,booktabs,array}
\usepackage{calc} % for calculating minipage widths
% Correct order of tables after \paragraph or \subparagraph
\usepackage{etoolbox}
\makeatletter
\patchcmd\longtable{\par}{\if@noskipsec\mbox{}\fi\par}{}{}
\makeatother
% Allow footnotes in longtable head/foot
\IfFileExists{footnotehyper.sty}{\usepackage{footnotehyper}}{\usepackage{footnote}}
\makesavenoteenv{longtable}
\usepackage{graphicx}
\makeatletter
\def\maxwidth{\ifdim\Gin@nat@width>\linewidth\linewidth\else\Gin@nat@width\fi}
\def\maxheight{\ifdim\Gin@nat@height>\textheight\textheight\else\Gin@nat@height\fi}
\makeatother
% Scale images if necessary, so that they will not overflow the page
% margins by default, and it is still possible to overwrite the defaults
% using explicit options in \includegraphics[width, height, ...]{}
\setkeys{Gin}{width=\maxwidth,height=\maxheight,keepaspectratio}
% Set default figure placement to htbp
\makeatletter
\def\fps@figure{htbp}
\makeatother
\setlength{\emergencystretch}{3em} % prevent overfull lines
\providecommand{\tightlist}{%
  \setlength{\itemsep}{0pt}\setlength{\parskip}{0pt}}
\setcounter{secnumdepth}{5}
\usepackage{booktabs}
\ifluatex
  \usepackage{selnolig}  % disable illegal ligatures
\fi
\usepackage[]{natbib}
\bibliographystyle{apalike}

\title{Introducción a la Ciencia de los Datos}
\author{Rubén Pizarro Gurrola}
\date{2021-03-20}

\begin{document}
\maketitle

{
\setcounter{tocdepth}{1}
\tableofcontents
}
\begin{center}\rule{0.5\linewidth}{0.5pt}\end{center}

\hypertarget{presentaciuxf3n}{%
\chapter{Presentación}\label{presentaciuxf3n}}

\includegraphics{images/instructores.jpg}

\hypertarget{instructores}{%
\section{Instructores}\label{instructores}}

\begin{verbatim}
MTI. José Gabriel Rodrígue Rivas, LI.

MTI. Marco Antonio Rodrígue Zúñiga, ISC.
\end{verbatim}

MAI. Rubén Pizarro Gurrola, LI.

Docente e investigador de tiempo completo del Instituto Tecnológico de Durango.

\begin{itemize}
\item
  Master en Business Intelligent y Data Science 2019-2020. Actualmente cursando en Innovation \& Entrepreneurship Business School S.L. IEBS.
\item
  Master en Ciencia de los Datos 2017. Big Data 4Success S.L.U; SoyData.Net
\item
  Actualmente estudios en Doctorado en Informática Universidad Americana de Europa UNADE.
\item
  Maestría en Administración de sistemas de Información en el Instituto Tecnológico y de Estudios Superiores de Monterrey (ITESM) campus Monterrey en 1995.
\item
  Licenciado en informática egresado del Instituto Tecnológico de Durango en 1991
\item
  Coordinador del libro ``Ciencia de los Datos. Propuestas y casos de uso''" con registro ISBN con número 978-607-8730-10-0.
\item
  Reconocimiento de perfil deseable registro;
\item
  Responsable del Cuerpo Académico en Formación de Desarrollo de Tecnología de Información basada en Ciencia de los Datos ITDUR-CA-14.
\item
  Líder de línea de investigación en ``Ciencia de los Datos''; miembro de la Red Colaborativa de Tecnologías de la información en Durango de COCYTED;
\item
  Miembro activo de la Red Iberoamericana de Academias de Investigación A.C. REDIBAI;
\item
  Miembro de la Red de Investigadores educativos REDIE en Durango, 2019.
\end{itemize}

\hypertarget{propuxf3sito}{%
\section{Propósito}\label{propuxf3sito}}

Este libro digital que se presenta y describe a continuación, es el resultado y consecuencia de la impartición del diplomado denominado ``\textbf{Ciencia de los datos e internet de las cosas}'' ofrecido por el \href{https://www.itdurango.edu.mx/}{Instituto Tecnológico de Durango} perteneciente al \href{https://www.tecnm.mx/}{Tecnológico Nacional de México}.

El objetivo de el diplomado fué el desarrollar habilidades en los participante en torno al paradigma industrias 4.0 relacionadas con ciencia de los datos e internet de las cosas IoT con el uso de herramientas adecuadas.

El diplomado constó de seis módulos:

\begin{enumerate}
\def\labelenumi{\arabic{enumi}.}
\item
  Modulo I. Introducción a la Ciencia de los datos
\item
  Modulo II. Bases de datos NoSQL
\item
  Modulo III. Fundamentos de programación en Python
\item
  Modulo IV. Internet de las cosas
\item
  Modulo V. Machine Learning en R
\item
  Modulo VI. Big data \& Analytics con Python
\end{enumerate}

Este libro digital es el resultado del ``\textbf{Módulo I}''\textbf{Introducción a la ciencia de los datos}", describe prácticas, ejercicios y desafíos realizados.

Se busca con este libro digital que el participante y lector, desarrollen habilidades de programación en lenguaje R, a través de R Studio para el análisis de datos estructurados y semiestructurados y su correcta interpretación y adecuada comunicación.

De manera específica se pretende lograr lo siguiente: instalar, configurar y organizar el entorno de trabajo con R y R Studio, realizar programas en R, utilizar técnicas descriptivas para analizar datos estructurados, analizar y evaluar datos semiestructurados, construir documentos markdown, publicar documentos en la nube por medio del servicio RPubs de R Studio y realizar aplicaciones WEB interactivas a través del servicio R Shiny, entre otros.

\hypertarget{intro}{%
\chapter{Introducción}\label{intro}}

\hypertarget{quuxe9-es-ciencia-de-los-datos}{%
\section{¿Qué es ciencia de los datos?}\label{quuxe9-es-ciencia-de-los-datos}}

Los datos y la información en las empresas son un activo necesario para la acertada toma de decisiones. Hoy en día y la gran cantidad de datos, la diversidad de formatos y la velocidad que se crean, requiere del uso de herramientas y tecnología para su rápido almacenamiento, procesamiento, análisis e interpretación veraz y oportuna, con ello obtener valor y conocimiento con la finalidad de realizar acciones que busquen la eficiencia y productividad en las organizaciones.

La información proviene de todos lados, sensores que reciben señales y se convierten en datos, publicaciones en las redes sociales, imágenes y vídeos digitales, registros de compra y transacciones, señales de GPS de los móviles, datos de las aplicaciones WEB, además de los tradicionales medios de comunicación como radio, televisión, prensa entre otros.

La ciencia de los datos como eje central, es un campo multidisciplinario que involucra los procesos y sistemas para extraer el conocimiento o un mejor entendimiento de grandes volúmenes de datos y sus diferentes formas estructurados y no estructurados. Pretende aprovechar aspectos tales como base de datos, tecnología \emph{big data}, aprendizaje automático (\emph{machine learning}) e, internet de las cosas (\emph{internet of things} (IoT)).

A pesar de que el término big data se asocia principalmente con cantidades de datos exorbitantes, se debe dejar de lado esta percepción, big data no va dirigido solo a gran tamaño, sino que abarca tanto volumen como variedad de datos y velocidad de acceso y procesamiento; además la Tecnología de Información y Comunicaciones (TIC) propicia que una gran cantidad se datos, estos deben~ procesarse, entenderse y transformarse en decisiones de valor, esto es el reto del big data.

\hypertarget{machine-learning}{%
\section{Machine learning}\label{machine-learning}}

El concepto de machine learning en la ciencia de los datos, es una área de la Inteligencia Artificial que engloba un conjunto de tareas, técnicas y algoritmos que hacen posible el aprendizaje de las computadoras sin intervención del humano a través del entrenamiento con grandes volúmenes de datos.

\hypertarget{internet-de-las-cosas}{%
\section{Internet de las cosas}\label{internet-de-las-cosas}}

Por otra parte, en un futuro cercano cualquier objeto cotidiano estará dotado de algún tipo de sensor que enviará información. El internet de las cosas (IoT) en la Ciencia de los Datos, está generando volúmenes masivos de datos estructurados y no estructurados.

\hypertarget{cientuxedfico-de-datos}{%
\section{Científico de datos}\label{cientuxedfico-de-datos}}

Las personas con las competencias adquiridas en el uso de éstas y otras herramientas y tecnologías, se dotan con ciertas características que dan lugar al surgimiento de un nuevo perfil profesional, el científico de datos (Data Scientist), que serían las personas capacitadas que deben saber de procesos, de tecnologías, del análisis e interpretación estadística, de comunicación de negocios, entre otros atributos.

El perfil del científico de datos es un ejemplo de la evolución de las profesiones que hacen uso de la información con el uso de la tecnología de información y comunicaciones (TIC), un híbrido entre un programador, analista, comunicador y consejero. Se trata de un profesional dedicado a analizar e interpretar grandes bases de datos.

\hypertarget{lenguaje-de-programaciuxf3n-r}{%
\section{Lenguaje de programación R}\label{lenguaje-de-programaciuxf3n-r}}

Es un lenguaje de programación para efectuar análisis de datos estadísticos y visualizar gráficas de los mismos datos. Además, es un software libre, gratuito, accesible y siempre a la vanguardia.

Descargar lenguaje de programamción R. \href{https://www.r-project.org/}{R Project}

\hypertarget{r-studio}{%
\section{R Studio}\label{r-studio}}

RStudio es un entorno de desarrollo integrado (IDE) para R. Incluye una consola, editor de resaltado de sintaxis que admite la ejecución directa de código, así como herramientas para el trazado, el historial, la depuración y la gestión del espacio de trabajo.

Descargar \href{https://rstudio.com/products/rstudio/}{R Studio}

\hypertarget{instalaciuxf3n-y-configuraciuxf3n-de-entorno}{%
\chapter{Instalación y configuración de entorno}\label{instalaciuxf3n-y-configuraciuxf3n-de-entorno}}

Este apartado describe el proceso para descargar e instalar las herramientas de trabajo para realizar práticas de programamción en lenguaje R.

Lenguaje R

R es un lenguaje y entorno para gráficos y computación estadística. Es de código abierto o libre de la comunidad y movimiento de software libre GNU.

R proporciona una amplia variedad de técnicas estadísticas (modelado lineal y no lineal, pruebas estadísticas clásicas, análisis de series de tiempo, clasificación, agrupamiento,\ldots) y técnicas gráficas, y es altamente extensible.

Una de las fortalezas de R es la facilidad con la que se pueden producir gráficos con calidad de publicación bien diseñados, incluidos símbolos matemáticos y fórmulas cuando sea necesario.

R es utilizado normalmente en contexto de investigación y docencia sin, embargo. existe una visión de su aplicación y uso en el ámbito empresarial.

R está disponible como Software Libre bajo los términos de la Licencia Pública General GNU de la Free Software Foundation en forma de código fuente \citep{r.fundation2021a}

Se puede \href{https://cran.itam.mx/}{descargar R de servidor itam.mx} de entre varios servidores o espejos (mirrors) disponibles para su descarga y elegir la distribución conforme y de acuerdo a las características de su equipo.

\begin{figure}
\centering
\includegraphics{images/descargar R.jpg}
\caption{Descargar R \citep{r.fundation2021}}
\end{figure}

Ahora bien

Descargar R

Instalar R

R Studio

Descarar R Studio

Instalar R Studio

\hypertarget{methods}{%
\chapter{Methods}\label{methods}}

We describe our methods in this chapter.

\hypertarget{applications}{%
\chapter{Applications}\label{applications}}

Some \emph{significant} applications are demonstrated in this chapter.

\hypertarget{example-one}{%
\section{Example one}\label{example-one}}

\hypertarget{example-two}{%
\section{Example two}\label{example-two}}

\hypertarget{final-words}{%
\chapter{Final Words}\label{final-words}}

We have finished a nice book.

  \bibliography{references.bib}

\end{document}
